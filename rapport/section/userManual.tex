\section{Manuel utilisateur}

\newpage

Pour ajouter des états normaux, composites, initiaux ou finaux il suffit de cliquer sur le bouton correspondant dans la barre de menu affichée en haut de l'application.


Les états sont déplaçables en les glissant/déposant quand le curseur de la souris indique que cela est possible.

Pour tracer une transition vers un autre état, il faut placer la souris au centre d'un état, celui-ci passe en surbrillance et à partir de ce moment il est possible de maintenir le clic gauche et tracer une transition vers un autre état.

Pour supprimer état ou transition, il faut sélectionner la cible et presser la touche Suppr. (ou sur Mac : fn + backspace).

Pour savoir si un graphe est valide, il convient de cliquer sur le bouton Validate qui informe l'utilisateur de la validité du graphe.

L'aplatissage d'un graphe se fait en pressant le bouton flatten. Si le graphe est invalide alors l'opération n'est pas effectuée.

Le double clic sur un état normal ou composite permet d'éditer son nom.

Concernant les transitions c'est pas encore FAIT.